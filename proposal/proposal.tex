\documentclass[preprint,11pt,3p]{article}

\usepackage{tocloft}
\usepackage{color}
\usepackage{hyperref}
\usepackage{graphicx}
\usepackage{float}
\usepackage{subcaption}
\usepackage{amsmath} 
\usepackage{tikz} 
\usepackage{epigraph}
\usepackage{lipsum} 
\usepackage{indentfirst}
\usepackage[strings]{underscore}

\usepackage{listings}
\usepackage{color}

\colorlet{light-gray}{gray!10}
\definecolor{javared}{rgb}{0.6,0,0} % for strings
\definecolor{javagreen}{rgb}{0.25,0.5,0.35} % comments
\definecolor{javapurple}{rgb}{0.5,0,0.35} % keywords
\definecolor{javadocblue}{rgb}{0.25,0.35,0.75} % javadoc
\definecolor{main-color}{rgb}{0.6627, 0.7176, 0.7764}
\definecolor{back-color}{rgb}{0.1686, 0.1686, 0.1686}
\definecolor{string-color}{rgb}{0.3333, 0.5254, 0.345}
\definecolor{key-color}{rgb}{0.8, 0.47, 0.196}
\definecolor{asparagus}{rgb}{0.53, 0.66, 0.42}
\definecolor{azure(colorwheel)}{rgb}{0.0, 0.5, 1.0}
\definecolor{ashgrey}{rgb}{0.7, 0.75, 0.71}

\definecolor{shadecolor}{RGB}{150,150,150}

\lstset{
  language=Java,
basicstyle=\small\ttfamily,
keywordstyle=\color{javapurple}\bfseries,
stringstyle=\color{javared},
    keywordstyle = {\color{javapurple}},
    keywordstyle = [2]{\color{asparagus}},
    keywordstyle = [3]{\color{azure(colorwheel)}},
    keywordstyle = [4]{\color{teal}},
    otherkeywords = {:,@@,|,->,>>=,val},
    morekeywords = [2]{;,:,*,@@},
    morekeywords = [3]{->,|},
    morekeywords = [4]{>>=},
commentstyle=\color{javagreen},
morecomment=[s][\color{javadocblue}]{(*}{*)},
numbers=left,
numberstyle=\tiny\color{black},
stepnumber=2,
numbersep=10pt,
tabsize=2,
showspaces=false,
showstringspaces=false,
escapeinside={(*@}{@*)},
% frame=single,
backgroundcolor=\color{light-gray},
frame=lines,
breaklines=true,
postbreak=\mbox{\textcolor{red}{$\hookrightarrow$}\space}}


\renewcommand\epigraphflush{flushright}
\renewcommand\epigraphsize{\normalsize}
\setlength\epigraphwidth{0.7\textwidth}
\renewcommand{\abstractname}{Executive Summary}

\definecolor{titlepagecolor}{cmyk}{1,.60,0,.40}

\DeclareFixedFont{\titlefont}{T1}{ppl}{b}{it}{0.5in}

\makeatletter                       
\def\printauthor{%                  
    {\large \@author}}              
\makeatother
\author{%
    Eric Altenburg \\
    \texttt{ealtenbu@stevens.edu}\vspace{20pt} \\
   	Daniel Kimball \\
    \texttt{dkimball@stevens.edu}\vspace{20pt} \\
    Hamzah Nizami \\
    \texttt{hnizami1@stevens.edu}\vspace{20pt} \\
    Max Shi \\
    \texttt{mshi7@stevens.edu}
    }

% The following code is borrowed from: https://tex.stackexchange.com/a/86310/10898

\newcommand\titlepagedecoration{%
\begin{tikzpicture}[remember picture,overlay,shorten >= -10pt]

	\coordinate (aux1) at ([yshift=-15pt]current page.north east);
	\coordinate (aux2) at ([yshift=-410pt]current page.north east);
	\coordinate (aux3) at ([xshift=-4.5cm]current page.north east);
	\coordinate (aux4) at ([yshift=-150pt]current page.north east);

	\begin{scope}[titlepagecolor!40,line width=12pt,rounded corners=12pt]
		\draw
		  (aux1) -- coordinate (a)
		  ++(225:5) --
		  ++(-45:5.1) coordinate (b);
		\draw[shorten <= -10pt]
		  (aux3) --
		  (a) --
		  (aux1);
		\draw[opacity=0.6,titlepagecolor,shorten <= -10pt]
		  (b) --
		  ++(225:2.2) --
		  ++(-45:2.2);
	\end{scope}
	\draw[titlepagecolor,line width=8pt,rounded corners=8pt,shorten <= -10pt]
	  (aux4) --
	  ++(225:0.8) --
	  ++(-45:0.8);
	\begin{scope}[titlepagecolor!70,line width=6pt,rounded corners=8pt]
		\draw[shorten <= -10pt]
		  (aux2) --
		  ++(225:3) coordinate[pos=0.45] (c) --
		  ++(-45:3.1);
		\draw
		  (aux2) --
		  (c) --
		  ++(135:2.5) --
		  ++(45:2.5) --
		  ++(-45:2.5) coordinate[pos=0.3] (d);   
		\draw 
		  (d) -- +(45:1);
	\end{scope}
\end{tikzpicture}%
}

\begin{document}
\begin{titlepage}

\noindent
\titlefont Lighthouse\par
% \epigraph{Piazza but better}%
\vspace*{1cm}
{\raggedright{{\textit{CS 546: Web Programming I |  Fall 2020}}}}
\null\vfill
\vspace*{1cm}
\noindent
\hfill
\begin{minipage}{0.35\linewidth}
    \begin{flushright}
        \printauthor
    \end{flushright}
\end{minipage}
%
\begin{minipage}{0.02\linewidth}
    \rule{1pt}{200pt}
\end{minipage}
\titlepagedecoration
\end{titlepage}




\newpage

\tableofcontents
\newpage

\section{Introduction}
We’ve all been there - it’s late at night and you’re diligently working on your CS546 programming lab. Everything is smooth sailing until you bump into your worst enemy: an ambiguous requirement. Would the function be expected to parse a float and return a result? Should the function be expected to handle a 7D array? What about an object nested within an object nested within an object? 

You turn to Slack, but it’s cluttered with too many things. The search feature is a bit clunky too if finding the best keyword to describe your concern is not your thing. Enter Lighthouse, an all in one website where you can go to find answers about all your assignment requirements details. With its smart auto-tagging features and sleek UI, you never have to search far to find the answer to your problem. 

\section{Motivation}
Reflecting on our time at Stevens, we’ve found that in order to run a successful course, communication among instructors and students is paramount. The team has used several platforms for course communication at Stevens: Slack, Piazza, Canvas discussions, and more. While all of them have their merits, they all fall into some familiar problems: too clunky, outdated UI, or hard to sort through. Lighthouse is our attempt at building a solution that is easy to use and quick in getting you the answer you need.

While platforms like Slack and Piazza are normally perfectly viable solutions, once the frequency of questions being asked increases, it becomes arduous to dig through all conversation threads and becomes much easier to simply ask a question without looking at previous threads, even if it might be a duplicate. What this ends up doing is it just adds to the sheer amount of conversation threads to sort through and makes it harder for the next person to find what they’re looking for. Thus, the cycle continues. Lighthouse aims to combat the problem of scale through auto-tagging features and identifying similar questions before submission and redirecting them to questions that might be able to solve their problem. 

\newpage
\section{Core Features}
\begin{enumerate}
	\item Landing Page
		\begin{enumerate}
			\item Sign up, login, and all of that jazz
			\item Brief explanation of the product
		\end{enumerate}
	\item Home Page
		\begin{enumerate}
			\item Has a list of all the courses you’re in
			\item Ability to add a course based on code and password
		\end{enumerate}
	\item Course Page
		\begin{enumerate}
			\item Specific to each individual course 
			\item Can filter questions/comments/announcements by tag
		\end{enumerate}
	\item Ask Question Page
		\begin{enumerate}
			\item Has a form where you can fill it out with a question and specify which class to ask it in 
			\item Before it submits, it checks to see how similar your question is to previous questions in that class and alerts you to other questions that may answer your question if it passes an arbitrary threshold of similarity  
		\end{enumerate}
	\item Questions Page
		\begin{enumerate}
			\item See a list of all the questions asked (course specific) 
			\item If you have the same question, upvote it so it gets higher priority! 
		\end{enumerate}
	\item Answer Page
		\begin{enumerate}
			\item Allows instructors and students to answer new questions 
			\item Instructors can endorse answers from students, just like in Piazza
		\end{enumerate}
\end{enumerate}	

\newpage
\section{Extra Features}
\begin{enumerate}
	\item Connect to Slack to filter through messages and auto-add them into the website
	\item Home page
		\begin{enumerate}
			\item An outline of your profile
		\end{enumerate}
	\item Course page
		\begin{enumerate}	
			\item Auto-tagged questions with user-defined tags through NLP features 
		\end{enumerate}	
	\item User profile
		\begin{enumerate}
			\item About me, what are you studying, courses enrolled in
				\begin{enumerate}
					\item Maybe some question/answer stats or rewards for answering questions or asking helpful questions
				\end{enumerate}
		\end{enumerate}
\end{enumerate}

\section{Github Repository}
\href{https://github.com/hniz/Lighthouse}{https://github.com/hniz/Lighthouse}

\end{document}


















