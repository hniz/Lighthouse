\documentclass[preprint,11pt,3p]{article}

\usepackage{tocloft}
\usepackage{color}
\usepackage{hyperref}
\usepackage{bera}
\usepackage{graphicx}
\usepackage{float}
\usepackage{subcaption}
\usepackage{amsmath} 
\usepackage{tikz} 
\usepackage{epigraph}
\usepackage{lipsum} 
\usepackage{indentfirst}
\usepackage[strings]{underscore}

\usepackage{listings}
\usepackage{color}

\colorlet{light-gray}{gray!10}
\definecolor{javared}{rgb}{0.6,0,0} % for strings
\definecolor{javagreen}{rgb}{0.25,0.5,0.35} % comments
\definecolor{javapurple}{rgb}{0.5,0,0.35} % keywords
\definecolor{javadocblue}{rgb}{0.25,0.35,0.75} % javadoc
\definecolor{main-color}{rgb}{0.6627, 0.7176, 0.7764}
\definecolor{back-color}{rgb}{0.1686, 0.1686, 0.1686}
\definecolor{string-color}{rgb}{0.3333, 0.5254, 0.345}
\definecolor{key-color}{rgb}{0.8, 0.47, 0.196}
\definecolor{asparagus}{rgb}{0.53, 0.66, 0.42}
\definecolor{azure(colorwheel)}{rgb}{0.0, 0.5, 1.0}
\definecolor{ashgrey}{rgb}{0.7, 0.75, 0.71}

\definecolor{shadecolor}{RGB}{150,150,150}

\lstset{
  language=Java,
basicstyle=\small\ttfamily,
keywordstyle=\color{javapurple}\bfseries,
stringstyle=\color{javapurple},
    keywordstyle = {\color{javapurple}},
    keywordstyle = [2]{\color{asparagus}},
    keywordstyle = [3]{\color{azure(colorwheel)}},
    keywordstyle = [4]{\color{teal}},
    otherkeywords = {:,@@,|,->,>>=,val},
    morekeywords = [2]{;,:,*,@@},
    morekeywords = [3]{->,|},
    morekeywords = [4]{>>=},
commentstyle=\color{javagreen},
morecomment=[s][\color{javadocblue}]{(*}{*)},
numbers=left,
numberstyle=\tiny\color{black},
stepnumber=1,
numbersep=10pt,
tabsize=2,
showspaces=false,
showstringspaces=false,
escapeinside={(*@}{@*)},
% frame=single,
backgroundcolor=\color{light-gray},
frame=lines,
breaklines=true,
postbreak=\mbox{\textcolor{red}{$\hookrightarrow$}\space}}


\renewcommand\epigraphflush{flushright}
\renewcommand\epigraphsize{\normalsize}
\setlength\epigraphwidth{0.7\textwidth}
\renewcommand{\abstractname}{Executive Summary}

\definecolor{titlepagecolor}{cmyk}{1,.60,0,.40}

\DeclareFixedFont{\titlefont}{T1}{ppl}{b}{it}{0.5in}

\makeatletter                       
\def\printauthor{%                  
    {\large \@author}}              
\makeatother
\author{%
    Eric Altenburg \\
    \texttt{ealtenbu@stevens.edu}\vspace{20pt} \\
   	Daniel Kimball \\
    \texttt{dkimball@stevens.edu}\vspace{20pt} \\
    Hamzah Nizami \\
    \texttt{hnizami1@stevens.edu}\vspace{20pt} \\
    Max Shi \\
    \texttt{mshi7@stevens.edu}
    }

% The following code is borrowed from: https://tex.stackexchange.com/a/86310/10898

\newcommand\titlepagedecoration{%
\begin{tikzpicture}[remember picture,overlay,shorten >= -10pt]

	\coordinate (aux1) at ([yshift=-15pt]current page.north east);
	\coordinate (aux2) at ([yshift=-410pt]current page.north east);
	\coordinate (aux3) at ([xshift=-4.5cm]current page.north east);
	\coordinate (aux4) at ([yshift=-150pt]current page.north east);

	\begin{scope}[titlepagecolor!40,line width=12pt,rounded corners=12pt]
		\draw
		  (aux1) -- coordinate (a)
		  ++(225:5) --
		  ++(-45:5.1) coordinate (b);
		\draw[shorten <= -10pt]
		  (aux3) --
		  (a) --
		  (aux1);
		\draw[opacity=0.6,titlepagecolor,shorten <= -10pt]
		  (b) --
		  ++(225:2.2) --
		  ++(-45:2.2);
	\end{scope}
	\draw[titlepagecolor,line width=8pt,rounded corners=8pt,shorten <= -10pt]
	  (aux4) --
	  ++(225:0.8) --
	  ++(-45:0.8);
	\begin{scope}[titlepagecolor!70,line width=6pt,rounded corners=8pt]
		\draw[shorten <= -10pt]
		  (aux2) --
		  ++(225:3) coordinate[pos=0.45] (c) --
		  ++(-45:3.1);
		\draw
		  (aux2) --
		  (c) --
		  ++(135:2.5) --
		  ++(45:2.5) --
		  ++(-45:2.5) coordinate[pos=0.3] (d);   
		\draw 
		  (d) -- +(45:1);
	\end{scope}
\end{tikzpicture}%
}

\begin{document}
\begin{titlepage}

\noindent
\titlefont Lighthouse\par
\epigraph{Database Proposal}%
% \vspace*{1cm}
{\textit{CS 546: Web Programming I |  Fall 2020}}
\null\vfill
\vspace*{1cm}
\noindent
\hfill
\begin{minipage}{0.35\linewidth}
    \begin{flushright}
        \printauthor
    \end{flushright}
\end{minipage}
%
\begin{minipage}{0.02\linewidth}
    \rule{1pt}{200pt}
\end{minipage}
\titlepagedecoration
\end{titlepage}




\newpage

\tableofcontents
\newpage

% So far for the DB:
% - 4 collections (users, posts, comments, classes)
% - users and posts will be connected to classes through a key reference
% - Comments will behave similarly to the "reviews" in the lab 6 with the movies where when you create a comment, you can add it to an array of inside of posts
% This is shaping up to be very similar similar to the lab 6 db format mixed with the lecture 7 code

% We might have to make a function that can update the list of classes the user has

\section{Users Collection}

\subsection{Description}
	The users collection will hold the data for each user including the _id, email, username, password, fullName, posts, type, comments under other users' posts, and a list of classes they are currently taking.

\subsection{Sample Users JSON}
\begin{lstlisting}
userCollection: [
	{
		"_id": ObjectId("7b7997a2-c0d2-4f8c-b27a-6a1d4b5b6310"),
		"email": "validEmail@address.com",
		"fullName": {"firstName": "Eric", "lastName": "Altenburg"},
		"posts": ["7b7997a2-c0d2-4f8c-b27a-6a1d4b5b6311", ... , "7b7997a2-c0d2-4f8c-b27a-6a1d4b5b63134"],
		"comments": ["7b7997a2-c0d2-4f8c-b27a-6a1d4b5b6411", ... , "7b7997a2-c0d2-4f8c-b27a-6a1d4b5b63434"],
		"classes": ["8b7997a2-c0d2-4f8c-b27a-6a1d4b5b6411", ... , "8b7997a2-c0d2-4f8c-b27a-6a1d4b5b63434"],
		"type": "instructor"
	},
	...,
	{
		"_id": ObjectId("7b7997a2-c0d2-4f8c-b27a-6a1d4b5b6510"),
		"email": "validEmail@address.com",
		"fullName": {"firstName": "John", "lastName": "Doe"},
		"posts": ["7b7997a2-c0d2-4f8c-b27a-6a1d4b5b6511", ... , "7b7997a2-c0d2-4f8c-b27a-6a1d4b5b63534"],
		"comments": ["8b7997a2-c0d2-4f8c-b27a-6a1d4b5b6511", ... , "8b7997a2-c0d2-4f8c-b27a-6a1d4b5b63534"],
		"classes": ["8b7997a2-c0d2-4f8c-b27a-6a1d4b5b6411", ... , "8b7997a2-c0d2-4f8c-b27a-6a1d4b5b63434"],
		"type": "student"
	}
]
\end{lstlisting}

\subsection{User Table}
\begingroup
\setlength{\tabcolsep}{15pt} % Default value: 6pt
\renewcommand{\arraystretch}{1.5} % Default value: 1
\begin{tabular}{| p{0.15\linewidth} | p{0.20\linewidth} | p{0.5\linewidth} |}
	\hline
	\textbf{Name} & \textbf{Type} & \textbf{Description} \\
	\hline
	_id & ObjectId & Automatically generated\\
	\hline
	email & String & Email address for the user, must be valid\\
	\hline
	fullName & Subdocument & Subdocument that contains the first and last name of the user which are both strings\\
	\hline
	posts & Array[String] & Array of post ids the user has posted\\
	\hline
	comments & Array[String] & Array of comment ids the user has commented on other posts\\
	\hline
	classes & Array[String] & Array of the classes the user is currently taking\\
	\hline
	type & String & String representing the type of account that this document represents\\
	\hline
\end{tabular}
\endgroup


\newpage

\section{Posts Collection}

\subsection{Description}
	The posts collection will hold the data for each post, which includes the _id, title, author, subtitle, time_submitted, content, class, tags, score, and a list of comments on this post.

\subsection{Sample Posts JSON}
\begin{lstlisting}
userCollection: [
	{
		"_id": ObjectId("7b7997a2-c0d2-4f8c-b27a-6a1d4b5b6310"),
		"title": "Question about Assignment 4",
		"author": "7b7997a2-c0d2-4f8c-b27a-6a1d4b5b6230",
		"content": "I have a question regarding the code...",
		"comments": ["7b7997a2-c0d2-4f8c-b27a-6a1d4b5b6411", ... , "7b7997a2-c0d2-4f8c-b27a-6a1d4b5b63434"],
		"score": 10,
		"time_submitted": "Sun, 20 Oct 2020 10:47:02 GMT",
		"tags": ["Assignment 4", "Question"]
	},
	...,
	{
		"_id": ObjectId("7b7997a2-c0d2-4f8c-b27a-6a1d4b5b6390"),
		"title": "Question about Assignment 2",
		"author": "7b7997a2-c0d2-4f8c-b27a-6a1d4b5b6230",
		"content": "I have a question regarding the wording of the assignment...",
		"comments": ["7b7997a2-c0d2-4f8c-b27a-6a1d4b5b641231", ... , "7b7997a2-c0d2-4f8c-b27a-6a11dfee163434"],
		"score": 10,
		"time_submitted": "Sun, 01 Nov 2020 20:47:02 GMT",
		"tags": ["Assignment 2", "Question"]
	}
]
\end{lstlisting}

\subsection{Posts Table}
\begingroup
\setlength{\tabcolsep}{15pt} % Default value: 6pt
\renewcommand{\arraystretch}{1.5} % Default value: 1
\begin{tabular}{| p{0.15\linewidth} | p{0.20\linewidth} | p{0.5\linewidth} |}
	\hline
	\textbf{Name} & \textbf{Type} & \textbf{Description} \\
	\hline
	_id & ObjectId & Automatically generated\\
	\hline
	title & String & Title of the post\\
	\hline
	author & String & The ID as a string of the author that references the author of the post\\
	\hline
	time_submitted & String & String representing the time that the post was submitted\\
	\hline
	content & String & The content of this post\\
	\hline
	comments & Array[String] & Array of comment ids for the comments on this post\\
	\hline
	tags & Array[String] & Array of associated tags with this post\\
	\hline
	score & Integer & The score of this post, as rated by peers\\
	\hline
\end{tabular}
\endgroup


\newpage

\section{Comments Collection}

\subsection{Description}
	The comments collection will hold the data for all comments users have posted on any posts in the Posts collection. This includes data such as _id, author, parent_post, time_submitted, and the content of the comment. 

\subsection{Sample Comments JSON}
\begin{lstlisting}
userCollection: [
	{
		"_id": ObjectId("7b7997a2-c0d2-4f8c-b27a-6a1d4b5b6310"),
		"author": "7b7997a2-c0d2-4f8c-b27a-103ab394c293",
		"parent_post": "7b7997a2-c0d2-4f8c-b27a-6a1d4b5b6310",
		"time_submitted": "Sun, 01 Nov 2020 10:27:02 GMT",
		"content": "I think you might be misinterpreting the problem."
	},
	...,
	{
		"_id": ObjectId("7b7997a2-c0d2-4f8c-b27a-6a1d4b5b6311"),
		"author": "7b7997a2-c0d2-4f8c-b27a-103ab394c2234",
		"parent_post": "7b7997a2-c0d2-4f8c-b27a-6a1d4b5b6310",
		"time_submitted": "Sun, 01 Nov 2020 12:27:42 GMT",
		"content": "No, I think he's right about that part."
	}
]
\end{lstlisting}

\subsection{Comments Table}
\begingroup
\setlength{\tabcolsep}{15pt} % Default value: 6pt
\renewcommand{\arraystretch}{1.5} % Default value: 1
\begin{tabular}{| p{0.15\linewidth} | p{0.20\linewidth} | p{0.5\linewidth} |}
	\hline
	\textbf{Name} & \textbf{Type} & \textbf{Description} \\
	\hline
	_id & ObjectId & Automatically generated\\
	\hline
	author & String & The ID as a string of the author that references the author of the post\\
	\hline
	parent_post & String & The ID as a string of the post that this comment is associated with\\
	\hline
	time_submitted & String & A string representing when the comment was posted\\
	\hline
	content & String & The content of the comment\\
	\hline
\end{tabular}
\endgroup


\newpage

\section{Classes Collection}

\subsection{Description}
	The classes collection will hold the data for each class in the system, which includes the _id, name, description, posts, tags, students, and instructor.

\subsection{Sample Classes JSON}
\begin{lstlisting}
userCollection: [
	{
		"_id": ObjectId("7b7997a2-c0d2-4f8c-b27a-6a1d4b5b6710"),
		"name": "CS 546 Web Programming I",
		"description": "Web Programming I course at Stevens Institute of Technology",
		"posts": ["7b7997a2-c0d2-4f8c-b27a-6a1d4b5b6311", ... , "7b7997a2-c0d2-4f8c-b27a-6a1d4b5b63134"],
		"students": ["7b7997a2-c0d2-4f8c-b27a-6a1d4b5b6411", ... , "7b7997a2-c0d2-4f8c-b27a-6a1d4b5b63434"],
		"tags": ["Question", "Clarification", "Assignment 1"],
		"instructor": "7b7997a2-c0d2-4f8c-b27a-6a1d4b5b6729",
	},
	...,
	{
		"_id": ObjectId("7b7997a2-c0d2-4f8c-b27a-6a1d4b5b6720"),
		"name": "CS 554 Web Programming II",
		"description": "Web Programming II course at Stevens Institute of Technology",
		"posts": ["7b7997a2-c0d2-4f8c-b27a-6a1d4b5b63123", ... , "7b7997a2-c0d2-4f8c-b27a-6a1d4b5b634567"],
		"students": ["7b7997a2-c0d2-4f8c-b27a-6a1d4b5b61231", ... , "7b7997a2-c0d2-4f8c-b27a-6a1d4b5b632354"],
		"tags": ["Question", "Clarification", "Assignment 1"],
		"instructor": "7b7997a2-c0d2-4f8c-b27a-6a1d4b5b6729",
	}
]
\end{lstlisting}


\newpage

\subsection{Classes Table}
\begingroup
\setlength{\tabcolsep}{15pt} % Default value: 6pt
\renewcommand{\arraystretch}{1.5} % Default value: 1
\begin{tabular}{| p{0.15\linewidth} | p{0.20\linewidth} | p{0.5\linewidth} |}
	\hline
	\textbf{Name} & \textbf{Type} & \textbf{Description} \\
	\hline
	_id & ObjectId & Automatically generated\\
	\hline
	name & String & Name of the course\\
	\hline
	description & String & Description of the course\\
	\hline
	posts & Array[String] & Array of post id associated with this class\\
	\hline
	students & Array[String] & Array of user ids for users enrolled in this class\\
	\hline
	tags & Array[String] & Array of strings for available tags on posts for this class\\
	\hline
	instructor & String & Id of instructor teaching this class\\
	\hline
\end{tabular}
\endgroup

\newpage

\section{Github Repository}
\href{https://github.com/hniz/Lighthouse}{https://github.com/hniz/Lighthouse}

\end{document}


















